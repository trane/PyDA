%        File: project.tex
%     Created: Mon Nov 07 08:00 PM 2011 M
% Last Change: Mon Nov 07 08:00 PM 2011 M
%
\documentclass[a4paper]{report}

\newcommand{\HRule}{\rule{\linewidth}{0.5mm}}
\begin{document}
\title{ % Upper part of the page
    \textsc{\LARGE CS 3100}\\[1.5cm]
    \textsc{\Large Final Project Report}\\[0.5cm]
    % Title
    \HRule \\[0.4cm]
    { \huge \bfseries PyDA - Push Down Automata}\\
    { \small https://github.com/trane/PyDA }\\
    \HRule \\[1.5cm]
}
\author{
    Landon Gilbert-Bland (gilbertb)\\ %u0262954\\
    Andrew Kuhnhausen (kuhnhaus)\\% u0275126\\
    Colton Myers (cmyers)\\% u0502549\\
}
\maketitle

\begin{abstract}
We have written a Python implementation of a Non-Deterministic Pushdown
Automaton called PyDA. For portability for other languages and the web, we have
defined our NPDA data structure using Javascript Object Notation (JSON). All
interaction with PyDA is through a Command Line Interface (CLI) which allows you
to load JSON encoded files, input test strings, step through the NPDA,
freeze/thaw threads as you step and print to both \texttt{.dot} and
\texttt{.pdf} files.  This imitates the functionality of JFLAP.
\end{abstract}
\end{document}

