%        File: project.tex
%     Created: Mon Nov 07 08:00 PM 2011 M
% Last Change: Mon Nov 07 08:00 PM 2011 M
%
\documentclass[a4paper]{report}
\usepackage{amsmath}
\usepackage{minted}

\newcommand{\HRule}{\rule{\linewidth}{0.5mm}}
\begin{document}
\title{ % Upper part of the page
    \textsc{\LARGE CS 3100}\\[1.5cm]
    \textsc{\Large Final Project Report}\\[0.5cm]
    % Title
    \HRule \\[0.4cm]
    { \huge \bfseries PyDA - Push Down Automata}\\
    { \small https://github.com/trane/PyDA }\\
    \HRule \\[1.5cm]
}
\author{
    Landon Gilbert-Bland (gilbertb)\\ %u0262954\\
    Andrew Kuhnhausen (kuhnhaus)\\% u0275126\\
    Colton Myers (cmyers)\\% u0502549\\
}
\maketitle

\begin{abstract}
We have written a Python implementation of a Non-Deterministic Pushdown
Automaton called PyDA. For portability for other languages and the web, we have
defined our NPDA data structure using Javascript Object Notation (JSON). All
interaction with PyDA is through a Command Line Interface (CLI) which allows you
to load JSON encoded files, input test strings, step through the NPDA,
freeze/thaw threads as you step and print to both \texttt{.dot} and
\texttt{.pdf} files.  This imitates the functionality of JFLAP.
\end{abstract}

\section*{Shared Tasks}
Much of the work done on PyDA was done together, including formally defining our
data structures, working out the flow from one module to another, and many other
small but significant decisions.
\subsection*{JSON Data Structure}
The reason we chose JSON as our data structure, is that it is language agnostic
(much like XML) but lends itself well for encoding the intended structure of a
PDA within the definition itself. The ability to define any PDA with JSON allows
anyone to write compatible software in any language and any platform. While JSON
is quite portable, there were some interesting limitations that complicated the
structure.
\subsubsection*{Validation at the Data Structure Level}
JSON allows us to do some low-level validation on the intended input by using
hashes to ensure that no duplicates are allowed where they shouldn't be. Since
each $\delta$ function must be of the form: $(p,a,A,q,\alpha)\in\delta$ and
those tuples must be unique, we found which elements would be repeated and
hashed them $(p, a, A, q)$. Then set $\alpha$ in an Array, since there can be
multiple push elements per hashed tuple $(p,a,A,q)$. The following is the output
of our $\delta$:
\begin{minted}{javascript}
"Delta": {
    "s0": { "@": { "Z": { "s1": ["SZ"] } } },
    "s1": { "a": { "a": { "s1": ["@"] } },
            "b": { "b": { "s1": ["@"] } },
            "c": { "c": { "s1": ["@"] } },
            "@": { "Z": { "s1": ["Z"] },
                   "I": { "s1": ["aK", "aJ"] },
                   "A": { "s1": ["aA", "B"] },
                   "K": { "s1": ["bJ"] },
                   "I": { "s1": ["aJ"] },
                   "B": { "s1": ["bB", "Bc", "@"] },
                   "S": { "s1": ["I", "C"] },
                   "J": { "s1": ["@", "bJc"] },
                   "C": { "s1": ["aaA", "B"] }
                 }
          }
}
\end{minted}
Notice that each element in $(p,a,A,q)$ is uniquely hashed, while $\alpha$ can
be one or more values to be pushed onto the stack.
\subsubsection*{The Final Structure}
For sets $Q, \Sigma, \Gamma, F$, we used Arrays and for single elements $q0, Z$
we used strings for the values of our hash. The final structure (asg7's L2 as an
example):
\begin{minted}{javascript}
{
    "q0": "s0",
    "F": [ "s2" ],
    "Q": [ "s0", "s1", "s2" ],
    "Delta": {
        "s0": { "@": { "Z": { "s1": ["SZ"] } } },
        "s1": { "a": { "a": { "s1": ["@"] } },
                "b": { "b": { "s1": ["@"] } },
                "c": { "c": { "s1": ["@"] } },
                "@": { "Z": { "s1": ["Z"] },
                       "I": { "s1": ["aK", "aJ"] },
                       "A": { "s1": ["aA", "B"] },
                       "K": { "s1": ["bJ"] },
                       "I": { "s1": ["aJ"] },
                       "B": { "s1": ["bB", "Bc", "@"] },
                       "S": { "s1": ["I", "C"] },
                       "J": { "s1": ["@", "bJc"] },
                       "C": { "s1": ["aaA", "B"] }
                     }
              }
    },
    "Z": "Z",
    "Sigma": [ "a", "b", "c" ],
    "Gamma": [ "c", "b", "a", "I", "K", "A", "J", "B", "S", "C", "Z" ]
}
\end{minted}

\subsection*{NPDA Data Structure}
The NPDA structure is based more on the functional, formal definition of a PDA.
Since Python allows structures like tuples, we are able to convert from JSON to
a more formal representation. $\delta$ is a tuple of $(p,a,A,q,\alpha)$,
$\Sigma, F, Q, \Gamma$ are sets and $q0, Z$ are single strings. Using asg7's L2
as an example, our Python data structure is:
\begin{minted}{python}
{
    'q0': 's0',
    'F': {'s2'},
    'Q': {'s2', 's1', 's0'},
    'Delta': {
        ('s1', '@', 'J', 's1', 'bJc'),
        ('s1', '@', 'K', 's1', 'bJ'),
        ('s1', '@', 'B', 's1', '@'),
        ('s1', 'b', 'b', 's1', '@'),
        ('s1', '@', 'B', 's1', 'bB'),
        ('s1', '@', 'B', 's1', 'Bc'),
        ('s1', 'a', 'a', 's1', '@'),
        ('s1', '@', 'A', 's1', 'B'),
        ('s0', '@', 'Z', 's1', 'SZ'),
        ('s1', 'c', 'c', 's1', '@'),
        ('s1', '@', 'A', 's1', 'aA'),
        ('s1', '@', 'C', 's1', 'B'),
        ('s1', '@', 'S', 's1', 'I'),
        ('s1', '@', 'C', 's1', 'aaA'),
        ('s1', '@', 'S', 's1', 'C'),
        ('s1', '@', 'Z', 's1', 'Z'),
        ('s1', '@', 'I', 's1', 'aJ'),
        ('s1', '@', 'J', 's1', '@')},
    'Z': 'Z',
    'Sigma': {'a', 'c', 'b'},
    'Gamma': {'a', 'A', 'c', 'b', 'I', 'K', 'J', 'C', 'S', 'B', 'Z'}
}
\end{minted}

\section*{Landon Gilbert-Bland}
\section*{Andrew Kuhnhausen}
I focused on the formal verification, conversion and CLI elements of the
assignment.
\subsection*{verify()}
\texttt{verify} ensures that $NPDA=(Q,\Sigma,\Gamma,\delta,q0,Z,F)$ is valid.
The following were asserted in the \texttt{verify()} method:
\begin{itemize}
    \item $\Sigma \ne \{\epsilon\}$
    \item $\epsilon \notin \Sigma$
    \item $q0\in Q$
    \item $F \subseteq Q$
    \item $Z \in \Gamma$
    \item $\{\forall a \in \delta | a \in \Sigma^{*}\}$
    \item $a\in\Sigma\cup\{\epsilon\}$
    \item $A\in\Gamma$
    \item $\alpha\in\Gamma^{*}$
    \item $Q\times(\Sigma\cup\{\epsilon\})\times\Gamma\rightarrow Q\times\Gamma^{*}$
\end{itemize}
\subsection*{normalize()}
\texttt{normalize()} is a mapping function from $JSON\rightarrow NPDA$. The
mappings are as follows:
\begin{itemize}
    \item $q0\rightarrow q0$
    \item $Z\rightarrow Z$
    \item $\{\forall x\in F \rightarrow F=\{x_0\cdots x_n\}\}$
    \item $\{\forall x\in Q \rightarrow Q=\{x_0\cdots x_n\}\}$
    \item $\{\forall x\in\Gamma \rightarrow \Gamma=\{x_0\cdots x_n\}\}$
    \item $\{\forall x\in\Sigma \rightarrow \Sigma=\{x_0\cdots x_n\}\}$
    \item $\{\\\forall p\in\delta \rightarrow
                \indent\forall a\in \{a\} \rightarrow
                \forall A\in \{A\} \rightarrow
                \forall q\in \{q\} \rightarrow
                \forall \alpha\in \{\alpha\} \rightarrow
                \{(p,a,A,q,a_0),\cdots,(p,a,A,q,a_n)\}\\\}$
\end{itemize}
\section*{Colton Myers}
\end{document}

