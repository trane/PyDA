%        File: project.tex
%     Created: Mon Nov 07 08:00 PM 2011 M
% Last Change: Mon Nov 07 08:00 PM 2011 M
%
\documentclass[a4paper]{report}

\newcommand{\HRule}{\rule{\linewidth}{0.5mm}}
\begin{document}
\title{ % Upper part of the page
    \textsc{\LARGE CS 3100}\\[1.5cm]
    \textsc{\Large Final Project Proposal}\\[0.5cm]
    % Title
    \HRule \\[0.4cm]
    { \huge \bfseries PyDA - Push Down Automata}\\
    \HRule \\[1.5cm]
}
\author{
    Landon Gilbert-Bland\\
    Andrew Kuhnhausen\\
    Colton Myers\\
}
\maketitle

Our group will be doing the following standard project:\\\\
\vbox{
Write Python routines to help experiment with non-deterministic push-down
automata (we will call them ``PDA'' for short). {\em Recall that we are studying
NPDAs; NPDA and DPDA {\bf are not equivalent} in the sense of language
recognition.} You must provide a file/module of function definitions (or use
Python classes) that maintain PDA configurations. Users must be able to input
PDAs, and an input string. Your program should allow users to pursue specific
configurations and move forward expanding the chosen configuration. At any
point, the user should be able to generate a PDF of their PDA with the
configuration shown below it. You can imitate functionality found in JFLAP {\em
with due credits given in your report}. All coding should be original work.
}
\end{document}

